\documentclass[]{article}
\usepackage{natbib}
\usepackage{graphicx}
\usepackage{amsmath}
\newcommand{\mbf}[1]{\mathbf{{#1}}}

\usepackage{todonotes}

% Document metadata
% =================
\newcommand{\Title}{
    Non-linear gravity inversion in spherical coordinates:
    application to the South American Moho
}
\newcommand{\Keywords}{
        Moho;
        gravity inversion;
        spherical coordinates;
        tesseroid;
        South America
}
\title{\Title}
\author{
    Leonardo Uieda \\ \texttt{leo@leouieda.com}
    \and
    Val\'eria C. F. Barbosa
}

\usepackage[pdftex,colorlinks=true,hidelinks]{hyperref}
\hypersetup{
    pdftitle={\Title},
    pdfauthor={Leonardo Uieda (leo@leouieda.com)},
    pdfsubject={},
    pdfkeywords={\Keywords},
    pdfcreator={pdfTeX}
}



\begin{document}

\maketitle

\noindent \textbf{Keywords:} \Keywords

\begin{abstract}
\end{abstract}

%%%%%%%%%%%%%%%%%%%%%%%%%%%%%%%%%%%%%%%%%%%%%%%%%%%%%%%%%%%%%%%%%%%%%%%%%%%%%%%
\section{Introduction}


%%%%%%%%%%%%%%%%%%%%%%%%%%%%%%%%%%%%%%%%%%%%%%%%%%%%%%%%%%%%%%%%%%%%%%%%%%%%%%%
\section{Methodology}

\subsection{Parametrization}

Remove the Normal Earth from observations to get disturbance.
Remove topography using Bouguer and correct crustal sources.
Leaves the relief of the Moho undulating around the Normal Earth Moho
(reference level).
Reference level is the Moho of the Normal Earth.

Assume all other sources were removed.
Only Moho topography around a reference level.
\todo{Include figure of Moho with all else  removed.}
Discretize into tesseroids.
Parameters are the Moho depths.
Establish that the reference level ($h_{ref}$)  and density contrast
($\Delta\rho_j$) are hyper-parameters.

\subsection{Inverse problem}

Formulate regularized inversion using Gauss-Newton and Steepest Descent.
Use full Jacobian.

\subsection{Bott's method}

Bott as Gauss-Newton.
Jacobian is diagonal.
Approximate the Jacobian by making tesseroid size tend to infinity to get
Bouguer slab.
This approximation is good for tesseroids because each will cover a large
enough area.
Show case for the Steepest Descent.
Extend the Jacobian to include lateral density variation.
Important because density-contrast is positive or negative depending if the
Moho is above or below the reference level.
Enforce that the use of a particular

Biggest computation time is in forward modeling.
All matrices involved are sparse.
Solve the sparse linear system with a conjugate gradient method.
\todo{Check which conjugate gradient scipy.space.linalg.solve uses.}
Use the regularized inversion because multiplying and solving the linear system
are fast compared to forward modeling.
Not using step size optimization because forward modeling is very slow so
increased convergence doesn't justify the extra time spent on the discarded
step attempts.

\subsection{Determination of hyper-parameters}



\subsection{Software implementation}

\todo{Cite packages in the references.txt file.}

\subsection{Satellite gravity data}


Explain where data came from and what was done to  it.


%%%%%%%%%%%%%%%%%%%%%%%%%%%%%%%%%%%%%%%%%%%%%%%%%%%%%%%%%%%%%%%%%%%%%%%%%%%%%%%
\section{Application to synthetic data from a simple model}

%\begin{figure}
    %\centering
    %\includegraphics[width=\columnwidth]{figures/figure}
    %\caption{This is a figure caption.}
    %\label{fig:myfig}
%\end{figure}

%%%%%%%%%%%%%%%%%%%%%%%%%%%%%%%%%%%%%%%%%%%%%%%%%%%%%%%%%%%%%%%%%%%%%%%%%%%%%%%
\section{Application to synthetic data from the CRUST1.0 model}



%%%%%%%%%%%%%%%%%%%%%%%%%%%%%%%%%%%%%%%%%%%%%%%%%%%%%%%%%%%%%%%%%%%%%%%%%%%%%%%
\section{Application to the South American Moho}



%%%%%%%%%%%%%%%%%%%%%%%%%%%%%%%%%%%%%%%%%%%%%%%%%%%%%%%%%%%%%%%%%%%%%%%%%%%%%%%
\section{Discussion}


%%%%%%%%%%%%%%%%%%%%%%%%%%%%%%%%%%%%%%%%%%%%%%%%%%%%%%%%%%%%%%%%%%%%%%%%%%%%%%%
\section{Conclusions}


%%%%%%%%%%%%%%%%%%%%%%%%%%%%%%%%%%%%%%%%%%%%%%%%%%%%%%%%%%%%%%%%%%%%%%%%%%%%%%%
\section{Acknowledgments}


\bibliographystyle{gji}
\bibliography{biblio}

\newpage
\listoftodos[Notes]

\end{document}
